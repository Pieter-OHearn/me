\documentclass[10pt, a4paper]{article}

% Packages:
\usepackage[
    ignoreheadfoot, % set margins without considering header and footer
    top=2 cm, % seperation between body and page edge from the top
    bottom=2 cm, % seperation between body and page edge from the bottom
    left=2 cm, % seperation between body and page edge from the left
    right=2 cm, % seperation between body and page edge from the right
    footskip=1.0 cm, % seperation between body and footer
    % showframe % for debugging 
]{geometry} % for adjusting page geometry
\usepackage[explicit]{titlesec} % for customizing section titles
\usepackage{tabularx} % for making tables with fixed width columns
\usepackage{array} % tabularx requires this
\usepackage[dvipsnames]{xcolor} % for coloring text
\definecolor{primaryColor}{RGB}{0, 150, 137} % define primary color
\usepackage{enumitem} % for customizing lists
\usepackage{fontawesome5} % for using icons
\usepackage{amsmath} % for math
\usepackage[
    pdftitle={Pieter OHearn's CV},
    pdfauthor={Pieter OHearn},
    colorlinks=true,
    urlcolor=primaryColor
]{hyperref} % for links, metadata and bookmarks
\usepackage[pscoord]{eso-pic} % for floating text on the page
\usepackage{calc} % for calculating lengths
\usepackage{bookmark} % for bookmarks
\usepackage{lastpage} % for getting the total number of pages
\usepackage{changepage} % for one column entries (adjustwidth environment)
\usepackage{paracol} % for two and three column entries
\usepackage{ifthen} % for conditional statements
\usepackage{needspace} % for avoiding page brake right after the section title
\usepackage{iftex} % check if engine is pdflatex, xetex or luatex
\usepackage{etoolbox} % for patchcmd

% Ensure that generate pdf is machine readable/ATS parsable:
\ifPDFTeX
    \input{glyphtounicode}
    \pdfgentounicode=1
    \usepackage[T1]{fontenc}
    \usepackage[utf8]{inputenc}
    \usepackage{lmodern}
\fi

\usepackage[default, type1]{sourcesanspro} 

% Some settings:
\AtBeginEnvironment{adjustwidth}{\partopsep0pt} % remove space before adjustwidth environment
\pagestyle{empty} % no header or footer
\setcounter{secnumdepth}{0} % no section numbering
\setlength{\parindent}{0pt} % no indentation
\setlength{\topskip}{0pt} % no top skip
\setlength{\columnsep}{0.15cm} % set column seperation
\makeatletter
\let\ps@customFooterStyle\ps@plain % Copy the plain style to customFooterStyle
\patchcmd{\ps@customFooterStyle}{\thepage}{
    \color{gray}\textit{\small Pieter OHearn - Page \thepage{} of \pageref*{LastPage}}
}{}{} % replace number by desired string
\makeatother
\pagestyle{customFooterStyle}

\titleformat{\section}{
    % avoid page braking right after the section title
    \needspace{4\baselineskip}
    % make the font size of the section title large and color it with the primary color
    \Large\color{primaryColor}
}{
}{
}{
    % print bold title, give 0.15 cm space and draw a line of 0.8 pt thickness
    % from the end of the title to the end of the body
    \textbf{#1}\hspace{0.15cm}\titlerule[0.8pt]\hspace{-0.1cm}
}[] % section title formatting

\titlespacing{\section}{
    % left space:
    -1pt
}{
    % top space:
    0.3 cm
}{
    % bottom space:
    0.2 cm
} % section title spacing

% \renewcommand\labelitemi{$\vcenter{\hbox{\small$\bullet$}}$} % custom bullet points
\newenvironment{highlights}{
    \begin{itemize}[
        topsep=0.10 cm,
        parsep=0.10 cm,
        partopsep=0pt,
        itemsep=0pt,
        leftmargin=0.4 cm + 10pt
    ]
}{
    \end{itemize}
} % new environment for highlights

\newenvironment{highlightsforbulletentries}{
    \begin{itemize}[
        topsep=0.10 cm,
        parsep=0.10 cm,
        partopsep=0pt,
        itemsep=0pt,
        leftmargin=10pt
    ]
}{
    \end{itemize}
} % new environment for highlights for bullet entries


\newenvironment{onecolentry}{
    \begin{adjustwidth}{
        0.2 cm + 0.00001 cm
    }{
        0.2 cm + 0.00001 cm
    }
}{
    \end{adjustwidth}
} % new environment for one column entries

\newenvironment{twocolentry}[2][]{
    \onecolentry
    \def\secondColumn{#2}
    \setcolumnwidth{\fill, 4.5 cm}
    \begin{paracol}{2}
}{
    \switchcolumn \raggedleft \secondColumn
    \end{paracol}
    \endonecolentry
} % new environment for two column entries

\newenvironment{threecolentry}[3][]{
    \onecolentry
    \def\thirdColumn{#3}
    \setcolumnwidth{1 cm, \fill, 4.5 cm}
    \begin{paracol}{3}
    {\raggedright #2} \switchcolumn
}{
    \switchcolumn \raggedleft \thirdColumn
    \end{paracol}
    \endonecolentry
} % new environment for three column entries

\newenvironment{header}{
    \setlength{\topsep}{0pt}\par\kern\topsep\centering\color{primaryColor}\linespread{1.5}
}{
    \par\kern\topsep
} % new environment for the header

\newcommand{\MonthName}[1]{%
    \ifcase#1%
        \or January%
        \or February%
        \or March%
        \or April%
        \or May%
        \or June%
        \or July%
        \or August%
        \or September%
        \or October%
        \or November%
        \or December%
    \else Unknown%
    \fi
}

\newcommand{\LastUpdatedText}{Last updated in \MonthName{\month} \the\year}

\newcommand{\placelastupdatedtext}{% \placetextbox{<horizontal pos>}{<vertical pos>}{<stuff>}
  \AddToShipoutPictureFG*{% Add <stuff> to current page foreground
    \put(
        \LenToUnit{\paperwidth-2 cm-0.2 cm+0.05cm},
        \LenToUnit{\paperheight-1.0 cm}
    ){\vtop{{\null}\makebox[0pt][c]{
        \small\color{gray}\textit{\LastUpdatedText}\hspace{\widthof{\LastUpdatedText}}
    }}}%
  }%
}%

% save the original href command in a new command:
\let\hrefWithoutArrow\href

% new command for external links:
\renewcommand{\href}[2]{\hrefWithoutArrow{#1}{\ifthenelse{\equal{#2}{}}{ }{#2 }\raisebox{.15ex}{\footnotesize \faExternalLink*}}}


\begin{document}
    \newcommand{\AND}{\unskip
        \cleaders\copy\ANDbox\hskip\wd\ANDbox
        \ignorespaces
    }
    \newsavebox\ANDbox
    \sbox\ANDbox{}

    \placelastupdatedtext
    \begin{header}
        \fontsize{30 pt}{30 pt}
        \textbf{Pieter OHearn}

        \vspace{0.3 cm}

        \normalsize
        \mbox{{\footnotesize\faMapMarker*}\hspace*{0.13cm}Amsterdam, Netherlands}%
        \kern 0.25 cm%
        \AND%
        \kern 0.25 cm%
        \mbox{\hrefWithoutArrow{mailto:pieter.ohearn@gmail.com}{{\footnotesize\faEnvelope[regular]}\hspace*{0.13cm}pieter.ohearn@gmail.com}}%
        \kern 0.25 cm%
        \AND%
        % \kern 0.25 cm%
        % \mbox{\hrefWithoutArrow{tel:+31612345678}{{\footnotesize\faPhone*}\hspace*{0.13cm}+31 06 12 34 56 78}}%
        % \kern 0.25 cm%
        % \AND%
        \kern 0.25 cm%
        \mbox{\hrefWithoutArrow{https://www.pieterohearn.com/}{{\footnotesize\faLink}\hspace*{0.13cm}pieterohearn.com}}%
        \kern 0.25 cm%
        \AND%
        \kern 0.25 cm%
        \mbox{\hrefWithoutArrow{https://linkedin.com/in/pieter-ohearn}{{\footnotesize\faLinkedinIn}\hspace*{0.13cm}linkedin.com/in/Pieter-OHearn}}%
        \kern 0.25 cm%
        \AND%
        \kern 0.25 cm%
        \mbox{\hrefWithoutArrow{https://github.com/pieter-ohearn}{{\footnotesize\faGithub}\hspace*{0.13cm}github.com/Pieter-OHearn}}%
    \end{header}

    \vspace{0.3 cm - 0.3 cm}

    \section{Summary}

        \begin{onecolentry}
            Senior Full-Stack Engineer specialising in identity and digital services platforms. Experienced in designing and scaling cross-platform solutions across Java/Spring, Node.js, and React Native ecosystems. Proven record of leading digital transformation initiatives in both public and private sectors, combining deep technical expertise with user-centric design. Passionate about observability, security-by-design, and clean architecture to deliver reliable, privacy-preserving digital products.
        \end{onecolentry}
    
    \section{Experience}

        \begin{twocolentry}{
            Utrecht, Netherlands
        
            Sep 2024 – Present
        }
            \textbf{Datakeeper}, Senior Full-Stack Engineer
            \begin{highlights}
                \item Implemented end-to-end observability with OpenTelemetry across the Spring Boot and Node.js back-ends for unified tracing and metrics.
                \item Re-architected the web scraping service for issuing verifiable credentials, improving resilience, privacy, and adaptability to external systems.
                \item Defining scalable back-end architectures in Spring and Node.js through close collaboration with product and compliance teams.
                \item Improved developer experience through clean architecture, modularisation, and automated testing.
            \end{highlights}
        \end{twocolentry}

        \vspace{0.2 cm}

        \begin{twocolentry}{
            Sydney, Australia

        Dec 2003 – Sep 2024
        }
            \textbf{Service NSW}, Senior Full-Stack Engineer | Mobile App Team
            \begin{highlights}
                \item Specialised in React Native mobile development across Android and iOS, maintaining and improving an app serving five million active users.
                \item Developed a component library, ensuring accessibility (WCAG 2.2 AA standard) and design consistency across all product teams, in close collaboration with UX and design.
                \item Implemented new citizen-facing features, such as a secure digital notification inbox and native transaction experiences, improving engagement and reducing friction.
                \item Designed and built a scalable Grants Service, integrating back-end APIs and mobile UI to help citizens apply for and track grants during crisis periods, increasing accessibility and reliability under high load.
                \item Partnered with designers, policy specialists, and engineers in a fast-moving, product-driven environment focused on impact, not process.
            \end{highlights}
        \end{twocolentry}

        \vspace{0.2 cm}

        \begin{twocolentry}{
            Sydney, Australia

        Mar 2022 – Dec 2023
        }
            \textbf{Service NSW}, Full-Stack Engineer | Digital Notifications Platform
            \begin{highlights}
                \item Re-architected a legacy monolith into microservices, introducing asynchronous queues and auto-scaling to ensure high availability during surges in message volume.
                \item Applied observability and performance monitoring tools, allowing the team to cut production incidents and improving API latency by 40–60%.
                \item Led a proof-of-concept database redesign, producing detailed ERDs and validating schema improvements through rigorous testing.
                \item Adopted the Backend-for-Frontend (BFF) pattern to deliver dynamically populated, data-driven front-end views that optimised the citizen experience.
            \end{highlights}
        \end{twocolentry}

        \vspace{0.2 cm}

        \begin{twocolentry}{
            Sydney, Australia

        Jun 2021 – Mar 2022
        }
            \textbf{Service NSW}, Junior Full-Stack Engineer | Digital Notifications Platform
            \begin{highlights}
                \item Developed and rigorously tested back-end APIs for a high-traffic notifications platform, facilitating communication between state agencies and over 8 million NSW citizens.
                \item Supported the design and deployment of front-end pages, focusing on improving user experience and accessibility.
            \end{highlights}
        \end{twocolentry}
    
    \section{Education}

        \begin{twocolentry}{
            Feb 2016 – Jul 2021
        }
            \textbf{The University of Sydney}, Computer Science
            \begin{highlights}
                \item Acquired a solid foundation in computer science theory, including computational complexity, operating systems, and computer architecture, providing a deep understanding of how software interacts with hardware.
                \item Developed practical skills in programming languages (e.g., Java, Python, C), databases, and web technologies, enabling me to build robust and versatile applications across various platforms.
            \end{highlights}
        \end{twocolentry}

        \begin{twocolentry}{
            Feb 2017 – Jul 2022
        }
            \textbf{The University of Sydney}, Software Engineering Honours
            \begin{highlights}
                \item Gained comprehensive knowledge in software engineering principles, including software development lifecycle, systems engineering, and project management, equipping me with the skills to lead and manage complex software projects.
                \item Specialised in advanced topics such as algorithms, data structures, and software design patterns, enhancing my ability to develop efficient, maintainable, and scalable software solutions.
            \end{highlights}
        \end{twocolentry}

    
    \section{Projects}
        
        \begin{twocolentry}{
            \href{https://github.com/pieter-ohearn/codekoala}{github.com/codekoala}
        }
            \textbf{AI Code Reviewing CLI Tool}
            \begin{highlights}
                \item Runs on-device AI to review git diffs, produce structured commit messages, and suggest code improvements.
                \item Tools Used: Python, CLI, Ollama (local LLM)
            \end{highlights}
        \end{twocolentry}

    
    \section{Technical Skills}

        \begin{itemize}[leftmargin=*, itemsep=0pt]
          \item Java; Kotlin; Spring Boot; Node.js; TypeScript; React; React Native; Angular; Python
          \item AWS; Docker; Terraform; CI/CD; Git;
          \item OpenTelemetry; Observability; Tracing; Sentry
          \item PostgreSQL; MongoDB; DynamoDB; Redis
          \item TDD; Clean Architecture; Agile/Scrum; Microservices; REST APIs
        \end{itemize}
    

\end{document}
